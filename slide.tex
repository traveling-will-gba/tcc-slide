\documentclass{beamer}

\usepackage{cmap}	
\usepackage{lmodern}	
\usepackage[T1]{fontenc}	
\usepackage[utf8]{inputenc}		
\usepackage{lastpage}		
\usepackage{indentfirst}
\usepackage{color}	
\usepackage{graphicx}	
\usepackage{units}
\usepackage[brazilian,hyperpageref]{backref}
\usepackage[alf]{abntex2cite}

\usepackage{bold-extra}
\usepackage{eso-pic}
\usepackage{xlop}
\usepackage{amsmath}
\usepackage{amsthm}
\usepackage{float}
\usepackage{tikz-cd}
\usepackage{mathrsfs} 
\usepackage{listings}
\usepackage{amsfonts}
\usepackage{diagbox}

\usepackage[linesnumbered,ruled]{algorithm2e}
\usepackage{pdflscape}
\usepackage{mathtools}
\usepackage{xcolor}% http://ctan.org/pkg/xcolor
\usepackage{color}
\usepackage{colortbl}
\usepackage{multirow}
\usepackage[english]{babel}
%\usepackage[utf8x]{inputenc}
% Choose how your presentation looks.
%
% For more themes, color themes and font themes, see:
% http://deic.uab.es/~iblanes/beamer_gallery/index_by_theme.html
%
\mode<presentation>
{
  \usetheme{default}      % or try Darmstadt, Madrid, Warsaw, ...
  \usecolortheme{default} % or try albatross, beaver, crane, ...
  \usefonttheme{default}  % or try serif, structurebold, ...
  \setbeamertemplate{navigation symbols}{}
  \setbeamertemplate{caption}[numbered]
} 

\DeclarePairedDelimiter{\ceil}{\lceil}{\rceil}%


\setbeamercovered{highly dynamic}
\newcounter{saveenumi}
\newcommand{\seti}{\setcounter{saveenumi}{\value{enumi}}}
\newcommand{\conti}{\setcounter{enumi}{\value{saveenumi}}}
\resetcounteronoverlays{saveenumi}

\title[Your Short Title]{Algoritmos para multiplicação rápida de dois números com mais de 1.000 dígitos}
\author{Arthur Luis Komatsu Aroeira}
\institute{Universidade de Brasilia - Faculdade Gama}
\date{06/12/2017}

\begin{document}

\begin{frame}
  \titlepage
\end{frame}

%Uncomment these lines for an automatically generated outline.
\begin{frame}{Outline}
 \tableofcontents
\end{frame}

\section{Introdução}

\begin{frame}{Introdução}

\begin{itemize}
  %\item 1
  %\item 2
  \item Computadores atuais conseguem realizar operações com números de tipicamente 64 bits, podendo representar $2^{64}=18446744073709551616$ valores distintos, números de até 19 dígitos
  \item Porém, há aplicações em que são necessários multiplicar números extremamente grandes (i.e. com mais de 1000 dígitos)
  \item Algumas aplicações importantes incluem Criptografia e Teoria dos Números
  \item Como solucionar esse problema?
\end{itemize}

%\vskip 1cm%

%\begin{block}{Examples}
%Some examples of commonly used commands and features are included, to help you get started.
%\end{block}

\end{frame}

\begin{frame}{Introdução}
	\begin{itemize}
		\item Solução: utilizar aritmética estendida.
		\item representar os números com seus dígitos gravados em posições distintas na memória (i.e. um array de dígitos)
		\item aplicar algoritmos para realizar operações com esses números
	\end{itemize}
\end{frame}

\section{Algoritmos}

\begin{frame}{Algoritmo da Multiplicação Tradicional}
	\begin{itemize}
		\item Algoritmo da Multiplicação Tradicional é o mais conhecido, comumente ensinado para crianças na escola desde cedo
		\item Consiste em aplicar deslocamentos e somas nos números
	\end{itemize}

	\begin{figure}[H]
		\centering
		\begin{center} \opmul{123}{456}\qquad \end{center}
		\caption{Procedimento da multiplicação tradicional para $123\times 456$}
		\label{fig:label}
	\end{figure}
\end{frame}

\begin{frame}{Algoritmo da Multiplicação Tradicional}

	\begin{itemize} 
		\item A ideia por trás deste algoritmo está em multiplicar polinômios com propagação de \textit{carrys}.\\
		\item O Algoritmo da Multiplicação Tradicional para $123\times456$ pode ser resumida nos seguintes passos:
	\end{itemize}

	\begin{enumerate}
		\item Expressar os números a serem multiplicados na representação em base decimal:
		\begin{equation}
			\begin{split}
				123&=1\cdot\textcolor{blue}{10^2}+2\cdot\textcolor{blue}{10^1}+3\cdot\textcolor{blue}{10^0}\\
				456&=4\cdot\textcolor{blue}{10^2}+5\cdot\textcolor{blue}{10^1}+6\cdot\textcolor{blue}{10^0}
			\end{split}
		\end{equation}
	\end{enumerate}

	\seti

\end{frame}

\begin{frame}{Algoritmo da Multiplicação Tradicional}
	\begin{enumerate}
		\conti
		\item Aplicar a multiplicação dos polinômios, alinhando as potências de 10 para facilitar a soma total

		\begin{figure}[H]
			\bgroup
			\def\arraystretch{1.2}%  1 is the default, change \omegahatever you need
			\begin{table}[H]
				\centering
				\begin{tabular}{cccccc}\label{mult}

				  	         & & & 1\textcolor{blue}{$\cdot 10^2$} & 2\textcolor{blue}{$\cdot 10^1$} & 3\textcolor{blue}{$\cdot 10^0$}    \\
					$\times$ & & & 4\textcolor{blue}{$\cdot 10^2$} & 5\textcolor{blue}{$\cdot 10^1$} & 6\textcolor{blue}{$\cdot 10^0$}    \\
					\hline
				  	         & & & 6\textcolor{blue}{$\cdot 10^2$} & 12\textcolor{blue}{$\cdot 10^1$} & 18\textcolor{blue}{$\cdot 10^0$}    \\
				  	           & & 5\textcolor{blue}{$\cdot 10^3$} & 10\textcolor{blue}{$\cdot 10^2$} & 15\textcolor{blue}{$\cdot 10^1$} &  \\
				  	+            &   4\textcolor{blue}{$\cdot 10^4$} & 8\textcolor{blue}{$\cdot 10^3$} & 12\textcolor{blue}{$\cdot 10^2$} && \\
					\hline
						     & 4\textcolor{blue}{$\cdot 10^4$} & 13\textcolor{blue}{$\cdot 10^3$} &28\textcolor{blue}{$\cdot 10^2$} &27\textcolor{blue}{$\cdot 10^1$}&18\textcolor{blue}{$\cdot 10^0$}\\
				\end{tabular}
				
			\end{table}
			\egroup
			\caption{Mutiplicação $123\times456$ vista como uma multiplicação de polinômios}
			\label{fig:mul}
		\end{figure}
		\seti
	\end{enumerate}
\end{frame}

\begin{frame}{Algoritmo da Multiplicação Tradicional}
	\begin{enumerate}
		\conti
		\item Aplicar a propagação de carrys

		\begin{figure}[H]
			\center
%			\begin{tikzcd}
%			 4   & \ar[bend left=40]{l}{1} 13 &\ar[bend left=40]{l}{3} 28 &\ar[bend left=40]{l}{2} 27 &\ar[bend left=40]{l}{1} 18 \\
%			 5 & 6 & 0 & 8 & 8
%			\end{tikzcd}
			\caption{Procedimento da propagação de carrys da multiplicação $123\times456$}
			\label{fig:propagacaocarrys}
		\end{figure}
		\seti
	\end{enumerate}

	\begin{itemize}
		\item A multiplicação de polinômios pode ser vista como uma convolução de seus elementos
		\item Logo, a multiplicação tradicional consiste basicamente em aplicar a operação de convolução nos dígitos dos números
	\end{itemize}
\end{frame}

\begin{frame}{Algoritmo da Multiplicação Tradicional}

	\begin{algorithm}[H]
		\SetKwInOut{Input}{entrada}\SetKwInOut{Output}{saída}
		\SetKwProg{Fn}{Function}{is}{end}

	    \Input{Dois vetores $a$ e $b$}
	    \Output{O vetor $c = a*b$}
	    \BlankLine

	    \Fn{Convolução $(a,b)$}
	    {
	    	$n \leftarrow sizeof(a)$\\
	    	$m \leftarrow sizeof(b)$\\
	    	$c[n + m - 1] \leftarrow (0, 0, ..., 0)$\\
		    \For{$i\leftarrow0$ \KwTo n-1}
		    {
		    	\For{$j\leftarrow0$ \KwTo m-1}
		    	{
		    		$c[i + j] \leftarrow c[i + j] + a[i] \cdot b[j]$
		    	}
		    }
		    return c
		}

	    \caption{Algoritmo de convolução de 2 vetores pela definição}
	    \label{alg:conv}
	\end{algorithm}
\end{frame}

\begin{frame}{Algoritmo da Multiplicação Tradicional}
	\begin{itemize}
		\item Complexidade do algoritmo de convolução: $O(nm)$
		\item Complexidade do algoritmo de propagação de \textit{carrys}: $O(n+m)$
		\item Complexidade total: $O(nm+n+m)=O(nm)$
		\item É o algoritmo mais simples de implementar, porém ineficaz para números com menos de 1000 dígitos
		\item É possível obter uma complexidade melhor?
	\end{itemize}
\end{frame}

\begin{frame}{Algoritmo de Karatsuba}
	\begin{itemize}
		\item Andrey Kolmogorov conjecturou em 1956 que o algoritmo tradicional era assintoticamente ótimo
		\item A história mudou em 1960, quando Kolmogorov anunciou sua conjectura em um de seus seminários e chamou atenção de um espectador: Anatoli Karatsuba.
		\item Após pensar ativamente sobre o assunto, Karatsuba desprovou a conjectura em menos de uma semana, encontrando um algoritmo mais rápido que o tradicional
	\end{itemize}
\end{frame}

\begin{frame}{Algoritmo de Karatsuba}
	\begin{itemize}
		\item O Algoritmo de Karatsuba utiliza uma técnica de dividir e conquistar
		\item O Algoritmo de Karatsuba para multiplicar dois números $A=a_{n-1}...a_1a_0$ e $B=b_{n-1}...b_1b_0$ consiste em aplicar os seguintes passos:
	\end{itemize}
	\begin{enumerate}
			\item Considerando $k = \ceil{n/2}$, escrever $A$ e $B$ como:

		\begin{equation}\label{eq:karatsuba}
			\begin{split}
		    	A &= A_0\cdot 10^{k}+A_1 \\
		    	B &= B_0\cdot 10^{k}+B_1\\
			\end{split}
		\end{equation}
		Isso equivale em dividir $A$ e $B$ pela metade de seus algarismos:

		\begin{equation}\label{eq:karatsuba2}
			\begin{split}
		    	A_0&= a_{n-1}...a_k\\
		    	A_1&= a_{k-1}...a_0\\
		    	B_0&= b_{n-1}...b_k\\
		    	B_1&= b_{k-1}...b_0\\
			\end{split}
		\end{equation}
		\seti
	\end{enumerate}
\end{frame}

\begin{frame}{Algoritmo de Karatsuba}
	\begin{enumerate}
		\conti
		\item Calculam-se três multiplicações: 
		\begin{equation}
			\begin{split}
		    	C_0 &= A_0 \cdot B_0 \\
		    	C_1 &= A_1 \cdot B_1 \\
		    	C_2 &= (A_0+B_0)\cdot (A_1+B_1) \\
	    	\end{split}
		\end{equation}

		\item Para combinar:

		\begin{equation}\label{eq:karatsuba3}
			\begin{split}
		    	C &= A\cdot B = (A_0\cdot 10^k+A_1) \cdot(B_0\cdot 10^k+B_1)\\
		    	C &= A_0B_0\cdot10^{2k}+(A_0B_1+A_1B_0)\cdot10^k + A_1B_1\\
		    	C &= A_0B_0\cdot10^{2k}+( (A_0+B_0)(A_1+B_1)-A_0A_1-B_0B_1 )\cdot10^k + A_1B_1\\
		    	C &= C_0\cdot10^{2k}+(C_2-C_1-C_0)\cdot10^k+C_1\\
			\end{split}
		\end{equation}
	\end{enumerate}
\end{frame}

\begin{frame}{Algoritmo de Karatsuba}
	\SetAlFnt{\small}
	\begin{algorithm}[H]
		\SetKwInOut{Input}{entrada}\SetKwInOut{Output}{saída}
		\SetKwProg{Fn}{Function}{is}{end}

	    \Input{Dois números $A = a_{n-1}...a_{1}a_{0}$ e $B = b_{n-1}...b_{1}b_{0}$ em base 10}
	    \Output{O número $C=A\cdot B=c_{2n-1}...c_1c_0$}
	    \BlankLine

	    \Fn{Karatsuba $(A, B)$ }
	    {
	    	$n \leftarrow max(sizeof(A), sizeof(B))$\\
	    	\If{$n<n_0$}{return MultiplicacaoTradicional($A$, $B$)}
	    	$k \leftarrow \ceil{n/2}$\\
	    	$A_0 \leftarrow (a_{n-1}, ..., a_{k})$\\
	    	$A_1 \leftarrow (a_{k-1}, ..., a_{0})$\\
	    	$B_0 \leftarrow (b_{n-1}, ..., b_{k})$\\
	    	$B_1 \leftarrow (b_{k-1}, ..., b_{0})$\\
	    	$C_0\leftarrow Karatsuba(A_0, B_0)$\\
	    	$C_1\leftarrow Karatsuba(A_1, B_1)$\\  
	    	$C_2\leftarrow Karatsuba(A_0+B_0, A_1+B_1)$\\
		    return $C_0 10^{2k}+(C_2-C_0-C_1)10^k+C_1$
		}
	    \caption{Algoritmo de Karatsuba em base 10}
	    \label{alg:karatsuba}
	\end{algorithm}
\end{frame}

\begin{frame}{Algoritmo de Karatsuba}
	\begin{itemize}
		\item Complexidade do algoritmo de Karatsuba: $O(n^{\log_{2}{3}})\approx O(n^{1.585})$
	\end{itemize}
\end{frame}

\begin{frame}
	\begin{itemize}
		\item 
	\end{itemize}
\end{frame}

\end{document}